%===============================================================================
% DOCUMENT
%===============================================================================

%% Document class
\documentclass[a4paper,12pt]{scrreprt}

%% Include packages
\usepackage{packages}

\begin{document}

%% Include custom commands
\include{commands}

\pagenumbering{gobble}

%% Build cover
\include{cover}
\makecover

%% Default geometry
\newgeometry{top=3cm,left=3cm,right=3cm,bottom=4cm}

%% Save default geometry
\savegeometry{default}

%% Load default geometry with:
% \loadgeometry{default}


%===============================================================================
% BEGIN ABSTRACT PAGE
%===============================================================================

\renewenvironment{abstract}
 {\par\noindent\textbf{\Large\abstractname}\par\bigskip}
 {}

\begin{flushleft}
\begin{abstract}
    As catástrofes naturais apresentam desafios significativos para a logística e distribuição
    de recursos essenciais em situações de emergência. 
    Este trabalho propõe uma abordagem baseada em algoritmos de procura para otimizar a distribuição
    alimentar em cenários de crise, minimizando o tempo de resposta e garantindo a eficiência no uso
    dos recursos disponíveis. Para isso, foi formulado um problema de procura em estado único, implementado
    por meio de um grafo que representa as localizações afetadas e suas conexões. A solução combina algoritmos
    clássicos de inteligência artificial com heurísticas desenvolvidas para considerar fatores como distância,
    tempo e urgência.
    \par \textbf{Área de Aplicação}: Inteligência Artificial, Logística, Distribuição de Recursos.
    \par \textbf{Palavras-Chave}: Algoritmos de Procura, Heurísticas, A*, Gulosa, BFS, DFS, Procura Uniforme.
\end{abstract}
\end{flushleft}

\pagebreak

%===============================================================================
% END ABSTRACT PAGE
%===============================================================================


%===============================================================================
% BEGIN INDEXES PAGES
%===============================================================================

%% Changes table of content name
\renewcommand{\contentsname}{Índice}
\renewcommand{\listfigurename}{Índice de Figuras}
\renewcommand{\listtablename}{Índice de Tabelas}

\tableofcontents
\pagebreak

\listoffigures
\pagebreak

\listoftables
\pagebreak


%===============================================================================
% END INDEXES PAGES
%===============================================================================

\pagenumbering{arabic}

%===============================================================================
% BEGIN DESCRIÇÃO DO PROBLEMA
%===============================================================================

\chapter{Descrição do Problema}

Este problema trata-se de um cenário de gestão de recursos em situações de emergência, onde é 
necessário coordenar a distribuição de suprimentos em zonas afetadas por desastres naturais.
O objetivo é otimizar a distribuição de recursos, minimizando o tempo de resposta e garantindo a
satisfação das necessidades das zonas afetadas.

O problema é composto por um conjunto de localizações, algumas das quais são zonas afetadas (e, de entre
estas, algumas com maior urgência). Para chegar a estas zonas, é necessário utilizar veículos que partem
de diferentes localizações. Existem diferentes tipos de veículos: mota, carro, camião, \textit{drone}, helicóptero,
avião, e barco. Cada veículo tem uma capacidade de carga, velocidade, capacidade de combustível e gasto de combustível
diferente. Como tal, deve ser feita uma gestão eficiente dos mesmos, de forma a que sejam utilizados de maneira
a otimizar as suas capacidades de transporte.

\begin{table}[ht]
    \centering
    \small % Reduce font size
    \setlength{\tabcolsep}{0.5pt} % Adjust column spacing
    \renewcommand{\arraystretch}{1.2} % Adjust row spacing
    \fontsize{9}{10}\selectfont
    \begin{tabular}{|c|c|c|c|c|c|}
        \hline
        \textbf{Veículo} & \textbf{Tipo} & \textbf{Capacidade (kg)} & \textbf{Velocidade (km/h)} & \textbf{Cap. Comb. (L)} & \textbf{Cons. Comb. (L/100km)} \\
        \hline
        Mota & Terrestre & 10 & 120 & 15 & 4 \\
        Carro & Terrestre & 500 & 160 & 50 & 8 \\
        Camião & Terrestre & 2000 & 100 & 300 & 30 \\
        \hline
        Drone & Aéreo & 5 & 100 & 5 & 2 \\
        Helicóptero & Aéreo & 700 & 250 & 300 & 70 \\
        Avião & Aéreo & 800 & 250 & 200 & 40 \\
        \hline
        Barco & Marítimo & 5000 & 50 & 1000 & 50 \\
        \hline
    \end{tabular}
    \caption{Características dos Veículos}
    \label{tab:veiculos}
\end{table}

Os suprimentos a distribuir são de diferentes tipos: alimentos, água e kits de primeiros socorros. Cada zona afetada
tem necessidades específicas, que devem ser satisfeitas com a distribuição dos suprimentos. A distribuição dos suprimentos
deve ser feita de forma a minimizar o tempo de resposta, garantindo que as zonas com maior urgência são atendidas primeiro.
De entre os suprimentos a distribuir, os alimentos são perecíveis, pelo que o seu tempo de entrega deve ser minimizado antes
que se estraguem.

%===============================================================================
% END DESCRIÇÃO DO PROBLEMA
%===============================================================================

%===============================================================================
% BEGIN FORMULAÇÃO DO PROBLEMA
%===============================================================================

\chapter{Formulação do Problema}

\begin{itemize}
    \item\textbf{Tipo:} Problema de Procura de Estado Único
    \item\textbf{Estado Inicial:} Todos os centros de distribuição, locais de abastecimento,
    veículos e suprimentos disponíveis são conhecidos.
     As localizações e necessidades das zonas afetadas são identificadas, com diferentes graus de urgência.
    \item\textbf{Estado Objetivo:} Todas as zonas afetadas são atendidas, com as suas necessidades satisfeitas, e os suprimentos
    são distribuídos de forma a minimizar o tempo de resposta e o desperdício alimentar.
        \item\textbf{Operadores (Ações Disponíveis):} 
        \begin{itemize}
                \item\textit{Mover Veículo:} Deslocar um veículo de uma zona para outra, respeitando as limitações de combustível e condições da rota.
                \item\textit{Carregar/Descarregar Suprimentos:} Realizar operações de carga e descarga nos veículos, levando em consideração as capacidades.
                \item\textit{Reabastecer Combustível:} Realizar paragens no percurso para reabastecimento dos veículos, tendo em conta o tempo
                necessário e a capacidade de combustível disponível.
        \end{itemize}
    \item\textbf{Custo da Solução:} \textcolor{red}{O custo é calculado com base na distância percorrida.}
    \end{itemize}

O grafo correspondente ao problema é composto por um conjunto de vértices, que representam as localizações, e um conjunto de arestas,
que representam as ligações entre as localizações. Cada aresta tem um peso associado, que corresponde à distância entre as localizações
que liga. Para o desenho dos grafos, foram utilizadas as bibliotecas \texttt{graphviz} e, alternativamente, \texttt{mathplotlib}.
Por conseguinte, o grafo inicial é o seguinte:

\begin{figure}[ht]
    \centering
    \includegraphics[width=0.8\textwidth]{img/graph.png}
    \caption{Grafo Inicial}
    \label{fig:grafo}
\end{figure}

%===============================================================================
% END FORMULAÇÃO DO PROBLEMA
%===============================================================================

%===============================================================================
% BEGIN TAREFAS REALIZADAS E DECISÕES TOMADAS
%===============================================================================

\chapter{Tarefas Realizadas e Decisões Tomadas}

\section{Implementação do Problema}

Para a implementação do problema, foram criadas as seguintes classes:

\begin{itemize}
    \item \textit{Vehicle:} Representa um veículo, com as suas características e capacidades.
    \item \textit{Catastrophe:} Possui as zonas afetadas, com as suas necessidades e graus de urgência.
    \item \textit{Supply:} Representa os suprimentos a distribuir - alimentos, água e kits de primeiros socorros -,
    com as suas características e tempos de validade.
    \item \textit{Simulation Data:} Contém os dados da simulação - locais com suprimentos, locais de abastecimento,
    veículos disponíveis, e zonas afetadas.
    \item \textit{Graph:} Contém o grafo com as localizações e ligações entre as mesmas.
    \item \textit{Mission Planner:} Classe principal, que gere a simulação.
\end{itemize}

\section{Implementação dos Operadores?}

\section{Algoritmos de Procura Utilizados}

Para a resolução do problema, foram utilizados os seguintes algoritmos de procura:

\begin{itemize}
    \item \textit{Breadth-First Search (BFS):} Procura em Largura, que expande todos os nós de um nível antes de passar para o próximo.
    \item \textit{Depth-First Search (DFS):} Procura em Profundidade, que expande um nó até ao limite antes de retroceder.
    \item \textit{Uniform-Cost Search (UCS):} Procura de Custo Uniforme, que expande o nó com menor custo de caminho.
    \item \textit{A* Search:} Procura A*, que combina a procura gulosa com a procura de custo uniforme, utilizando uma das heurísticas
    disponíveis.
    \item \textit{Greedy:} Procura Gulosa, que expande o nó que parece mais promissor, com base numa das heurísticas disponíveis.
\end{itemize}

\section{Heurísticas Desenvolvidas}

Foram desenvolvidas as seguintes heurísticas para a procura A* e Greedy:
\textcolor{red}{Descrever as heurísticas desenvolvidas - uma de distância apenas, uma de tempo e distância,
e uma de tempo, distância e urgência?}



%===============================================================================
% END TAREFAS REALIZADAS E DECISÕES TOMADAS
%===============================================================================

%===============================================================================
% BEGIN CONCLUSÕES E DISCUSSÃO DOS RESULTADOS OBTIDOS
%===============================================================================

\chapter{Conclusões e Discussão dos Resultados Obtidos}

\section{Resultados Obtidos}
Fazer tabelas ou graficos

\section{Conclusões}

O trabalho demonstrou a eficácia dos algoritmos de procura na resolução de problemas
de logística em cenários de emergência, com destaque para a capacidade do algoritmo A* 
de integrar múltiplos critérios de decisão. Os resultados mostram que a escolha apropriada
de heurísticas pode reduzir significativamente o tempo de resposta, sobretudo em cenários
com alta urgência.

%===============================================================================
% END CONCLUSÕES E DISCUSSÃO DOS RESULTADOS OBTIDOS
%===============================================================================



%===============================================================================
% BEGIN REFERÊNCIAS
%===============================================================================

%% Change bibliography name from “Bibliografia” to “Referências”
\renewcommand\bibname{Referências}

\begin{thebibliography}{9}

\bibitem{RussellNorvig}
Russell, S., \& Norvig, P. (2009). \textit{Artificial Intelligence - A Modern Approach} (3rd ed.). Pearson Education. ISBN-13: 9780136042594.

\bibitem{CostaSimoes}
Costa, E., \& Simões, A. (2008). \textit{Inteligência Artificial - Fundamentos e Aplicações}. FCA. ISBN: 978-972-722-34.

\end{thebibliography}

%% Add bibliografia to index
\addcontentsline{toc}{chapter}{Referências}

%===============================================================================
% END REFERÊNCIAS
%===============================================================================





\end{document}
