%===============================================================================
% DOCUMENT
%===============================================================================

%% Document class
\documentclass[a4paper,12pt]{scrreprt}

%% Include packages
\usepackage{packages}

\begin{document}

%% Include custom commands
\include{commands}

\pagenumbering{gobble}

%% Build cover
\include{cover}
\makecover

%% Default geometry
\newgeometry{top=3cm,left=3cm,right=3cm,bottom=4cm}

%% Save default geometry
\savegeometry{default}

%% Load default geometry with:
% \loadgeometry{default}


%===============================================================================
% BEGIN ABSTRACT PAGE
%===============================================================================

\renewenvironment{abstract}
 {\par\noindent\textbf{\Large\abstractname}\par\bigskip}
 {}

\begin{flushleft}
\begin{abstract}
    <<O resumo tem como objectivo descrever de forma sucinta o trabalho realizado. Deverá conter uma pequena introdução, seguida por uma breve descrição do trabalho realizado e terminando com uma indicação sumária do seu estado final.>>
    \par \textbf{Área de Aplicação}: <<Identificação da Área de trabalho.>>
    \par \textbf{Palavras-Chave}: <<Conjunto de palavras-chave que permitirão referenciar domínios de conhecimento, tecnologias, estratégias, etc., directa ou indirectamente referidos no relatório.>>
\end{abstract}
\end{flushleft}

\pagebreak

%===============================================================================
% END ABSTRACT PAGE
%===============================================================================


%===============================================================================
% BEGIN INDEXES PAGES
%===============================================================================

%% Changes table of content name
\renewcommand{\contentsname}{Índice}
\renewcommand{\listfigurename}{Índice de Figuras}
\renewcommand{\listtablename}{Índice de Tabelas}
\renewcommand{\lstlistlistingname}{Índice de \textit{Snippets}}

\tableofcontents
\pagebreak

\listoffigures
\pagebreak

\listoftables
\pagebreak

\lstlistoflistings
\pagebreak


%===============================================================================
% END INDEXES PAGES
%===============================================================================

\pagenumbering{arabic}

%===============================================================================
% BEGIN DESCRIÇÃO DO PROBLEMA
%===============================================================================

\chapter{Descrição do Problema}

%===============================================================================
% END DESCRIÇÃO DO PROBLEMA
%===============================================================================

%===============================================================================
% BEGIN FORMULAÇÃO DO PROBLEMA
%===============================================================================

\chapter{Formulação do Problema}

%===============================================================================
% END FORMULAÇÃO DO PROBLEMA
%===============================================================================

%===============================================================================
% BEGIN TAREFAS REALIZADAS E DECISÕES TOMADAS
%===============================================================================

\chapter{Tarefas Realizadas e Decisões Tomadas}

%===============================================================================
% END TAREFAS REALIZADAS E DECISÕES TOMADAS
%===============================================================================

%===============================================================================
% BEGIN CONCLUSÕES E DISCUSSÃO DOS RESULTADOS OBTIDOS
%===============================================================================

\chapter{Conclusões e Discussão dos Resultados Obtidos}

%===============================================================================
% END CONCLUSÕES E DISCUSSÃO DOS RESULTADOS OBTIDOS
%===============================================================================



%===============================================================================
% BEGIN REFERÊNCIAS
%===============================================================================

%% Change bibliography name from “Bibliografia” to “Referências”
\renewcommand\bibname{Referências}

\begin{thebibliography}{9}

\bibitem{RussellNorvig}
Russell, S., \& Norvig, P. (2009). \textit{Artificial Intelligence - A Modern Approach} (3rd ed.). Pearson Education. ISBN-13: 9780136042594.

\bibitem{CostaSimoes}
Costa, E., \& Simões, A. (2008). \textit{Inteligência Artificial - Fundamentos e Aplicações}. FCA. ISBN: 978-972-722-34.

\end{thebibliography}

%% Add bibliografia to index
\addcontentsline{toc}{chapter}{Referências}

%===============================================================================
% END REFERÊNCIAS
%===============================================================================





\end{document}
